

\documentclass{article}

\usepackage{multicol}
\setlength{\columnsep}{1.2cm}
\usepackage{pgfplots}
\pgfplotsset{compat = newest}
\usepackage{titlesec}
\usepackage{graphicx}
\usepackage{wrapfig}
\usepackage{amsfonts}
\usepackage{tikz}
\usepackage{amssymb}
\usepackage{amsfonts}
\usepackage{amsmath}


\begin{document}

%----------------------------------------------------------

%\newpage
\section{Quadratic Functions}

\textbf{Vertex Form: $y = a(x-h)^2+k$} , V(h,k) \\
\textbf{Standard Form: $y = ax^2+bx+c$} , (0, c) = y-int \\
\textbf{Factored Form: $y = a(x-s)(x-t)$} , (s,0),(t,0) = x-ints\\\\

\noindent
\textbf{First differences} - if the first diffs are the same, the function is linear \\
\textbf{Second differences} - if the second diffs are the same, the function is quadratic\\

%----------------------------------------------------------

%\newpage
\section{Properties of Quadratics}

\subsection*{Complete the Square}
\begin{enumerate}
    \item Factor a out of the first two terms
    \item Take the coefficient of the second term, divide by 2 and square it
    \item Take the answer from 2. and add/sub inside the brackets
    \item Factor the trinomial, move the negative outside and simplify
\end{enumerate}

\subsection*{Changing Form}
factored to standard : expand\\
standard to factored : factor it\\
vertex to standard : expand \\
standard to vertex : complete the square

\subsection*{Terminology}
(R) Revenue - total amount of income (R = pQ)\\
(C) Cost - cost to produce items sold \\
(p) price - how much you sell it for \\
(P) Profit - amount made after costs deducted (P = R - C)\\
(Q) Quantity - how many sold (usually x)\\


%----------------------------------------------------------

%\newpage
\section{Inverse of a Quadratic}

\subsection*{Inverting Functions}
1) Given a graph: Select key points, flip the coordinates, regraph OR reflection over y = x\\
2) Given an equation: let y = ... (remove func notation), switch x and y variables, solve for y, go back to function notation\\

*** To restict, set x to be greater than the axis of symmetry \\


%----------------------------------------------------------

%\newpage
\section{Radicals}

\textbf{Radical} - a square, cube, or higher root ($\sqrt{}$ - called the radical symbol)

\subsection*{Properties}
\noindent
\begin{enumerate}
    \item $\sqrt{a}\sqrt{b} = \sqrt{ab}$
    \item $a\sqrt{b} + c\sqrt{d} = ac\sqrt{bd}$
    \item $a\sqrt{b} + c\sqrt{b} = (a+c)\sqrt{b}$
    \item $\sqrt{a}\sqrt{a} = a$
    \item $\sqrt{\frac{a}{b}} = \frac{\sqrt{a}}{\sqrt{b}}$
\end{enumerate}

%----------------------------------------------------------

%\newpage
\section{Solving Quadratic Equations}

\subsection*{3.5 and 3.6}

\subsection*{The Quadratic Formula}
$x=\frac{-b\pm\sqrt{b^2-4ac}}{2a}$

\subsection*{The Discriminant}
$D = b^2-4ac$\\\\
\noindent
If D is positive - 2 real roots\\
If D is zero - 1 real root\\
If D is negative - no real roots\\
If D is perfect square - factorable\\


%----------------------------------------------------------

%\newpage
\section{Families of Quadratics}
Quadratics that share a common property.

%----------------------------------------------------------

%\newpage
\section{Linear Quadratic Systems}
The intersection of a line and a parabola\\\\

\noindent
\textbf{Secant} - 2 solutions\\
\textbf{Tangent} - 1 solution\\\\

\noindent
Question wording: 
a) Find the POI (use sub/elim)\\
b) If there is only one POI then what is... (solve this by setting D to 0)\\

%----------------------------------------------------------
%\newpage
\section{WORD PROBLEMS (extra section)}

https://dnsva.github.io/REVIEWS/MATH/math.html


%----------------------------------------------------------

\end{document}