

\documentclass{article}

\usepackage{multicol}
\setlength{\columnsep}{1.2cm}
\usepackage{pgfplots}
\pgfplotsset{compat = newest}
\usepackage{titlesec}
\usepackage{graphicx}
\usepackage{wrapfig}
\usepackage{amsfonts}
\usepackage{tikz}
\usepackage{amssymb}
\usepackage{amsfonts}
\usepackage{amsmath}


\begin{document}

%----------------------------------------------------------

%\newpage
\section{Adding/Subtracting/Multiplying Polynomials}

\textbf{Polynomial} - An algebraic expression containing two or more terms

\subsection*{Adding/Subtracting}
When adding or subtracting polynomials, we add/subtract "like" terms

\subsection*{Multiplying}
When multiplying polynomials...\\
- use distributive property (sometimes more than once) \\
- be careful with signs\\

\noindent
\textbf{Associative Property} - the way the factors are grouped in a multiplication problem does not change the product.


%----------------------------------------------------------

%\newpage
\section{Factoring Polynomials}

\subsection*{Greatest Common Factor (GCF)}
Divide out largest common coefficient and variable with max common exponent 

\subsection*{Grouping}
- Group terms & factor GCF from each group\\
- Factor common brackets if possible 

\subsection*{Simple Trinomial}
$x^2+(a+b)x+ab = (x+a)(x+b)$

\subsection*{Complex Trinomial}
$ax^2+bx+c, a \neq 1$\\
- Find factors of a \& place in each bracket\\
- Find factors of c \& place in each bracket\\
- Check by explanding in your head if it fits, if not, try something else

\subsection*{Difference of Squares}
$a^2-b^2 = (a-b)(a+b)$

\subsection*{Perfect Squares}
$a^2+2ab+b^2 = (a+b)^2$\\
$a^2-2ab+b^2 = (a-b)^2$

\subsection*{Sum and Difference of Cubes}
$a^3+b^3 = (a+b)(a^2-ab+b^2)$\\
$a^3-b^3 = (a-b)(a^2+ab+b^2)$

\subsection*{Long Division}
quotient*divisor = dividend\\


%----------------------------------------------------------

%\newpage
\section{Rational Expressions}

A rational expression is an expression that is the ratio of two polynomials,\\
\begin{center}
$f(x) = \frac{R(x)}{H(x)}$
\end{center}
When working with rational expressions, it is important to state \textbf{restrictions}. Restrictions are values of the variable that cause the function to be undefined.\\\\

\noindent
The two types of restrictions are \textbf{holes} and \textbf{asymptotes}. Restrictions arise from any factors that have ever appeared in the denominator.\\\\

\noindent
A \textbf{hole} is a missing value in the graph. It occurs when a restriction is \textbf{removed through simplification}.\\\\
A \textbf{asymptote} is a line that the graph approaches but never crosses. It occurs when a restriction \textbf{remains afer being simplified}.\\\\

\noindent
Note - 2 rational expressions are equivalent IF they are the same for all possible values of the domain.\\

%----------------------------------------------------------

%\newpage
\section{Multiplying and Dividng Rational Expressions}

\subsection*{Steps}
\begin{enumerate}
    \item Factor numerator and denominator
    \item Divide out common factors to simplify 
    \item Identify restrictions
\end{enumerate}

\noindent
Note - When dividing by fractions we multiply by the reciprocal, restrictions are determined by all factors that have ever appeared in the denominator\\
%----------------------------------------------------------

%\newpage
\section{Adding and Subtracting Rational Expressions}
Note: You cannot cross cancel when adding/subtracting, only when multiplying\\
Goal: Get a common denominator

\subsection*{Steps}
\begin{enumerate}
    \item Factor denominator
    \item Identify restrictions
    \item Find lowest common denominator
    \item Expres rational exp. with same LCD
    \item Add/subtract tops \& keep LCD
    \item Expland out and refactor numerator to see if refactorable
\end{enumerate}


%----------------------------------------------------------

%\newpage
\section{Graphs of Rational Functions}
Rational functions usually come out to be either linear (only holes no asymptotes) or reciprocal (can be both) functions\\
\subsection*{Steps}
\begin{enumerate}
    \item Simplify expression
    \item State and classify restrictions
    \item Draw function and add any holes
\end{enumerate}

%----------------------------------------------------------

\end{document}